\documentclass[10pt, a4paper, final]{article}

%%% PACKAGES and FORMATTING
\usepackage[margin=1.5cm, vcentering=true]{geometry}

\usepackage[english]{babel}
\usepackage[UKenglish]{isodate}
\cleanlookdateon

\usepackage[hidelinks, final]{hyperref}
\hypersetup{
  pdfauthor = {Ian Dexter D. Marquez},
  pdftitle = {Resume - Ian Dexter D. Marquez},
  pdfsubject = {Generated using Python and LaTeX},
  pdfkeywords = {Amazon Web Services, Ansible, Apache, Application deployment and support, Bash/Shell, CentOS, Cisco IOS, Courier, Debian, DevOps, Dokuwiki, Dovecot, Drupal, Ethereal/Wireshark, F5 BIG-IP ZebOS, Fedora, Google App Engine, IBM AIX, IaaS, Identity management, Infrastructure automation, Javascript, MediaWiki, Microsoft Windows 2003 Server, MySQL, Nagios, NetCool, Network design, development and administration, OpenShift, OpenStack, Oracle, PHP, PaaS, Perl, Postfix, PostgreSQL, Procmail, Puppet, Python, Red Hat Enterprise Linux, SQL, Sendmail, Sniffer, Squid, SuSE Linux Enterprise Server, Subversion, Sun Directory Server, Sun Identity Manager, Sun Solaris, Sybase, TWiki, Tomcat, Trend Micro InterScan Web and Messaging Security Suite for Windows and Unix, Ubuntu, Unix/Linux systems administration, VMware, VirtualBox, Web design, development and management, WordPress, git, tcpdump, resume, LaTeX, Jinja2}
}

\usepackage[osf]{libertine}
\renewcommand*\familydefault{\sfdefault}
\usepackage[T1]{fontenc}

\usepackage{titlesec}
\titleformat{\section}{\sc\large\raggedright}{}{0em}{}
\titleformat{\subsection}{\bf\raggedright}{}{0em}{}
\titlespacing{\section}{0pt}{5ex}{1ex}
\titlespacing{\subsection}{0pt}{0.5ex}{0ex}

\usepackage{paralist}
\usepackage{setspace}

\usepackage[usenames,dvipsnames]{xcolor}

\usepackage{fancyhdr}
\fancyhf{}
\renewcommand{\headrulewidth}{0pt}
\renewcommand{\footrulewidth}{0.1pt}
\renewcommand{\footrule}{\hbox to\headwidth{%
  \color{gray}\leaders\hrule height \footrulewidth\hfill}}
\fancyfoot[L]{\small \textcolor{gray}{Last update: \textsc{\today}}}

\pagestyle{fancy}
\setlength{\parskip}{0cm}
\setlength{\parindent}{0cm}
%%% end - PACKAGES and FORMATTING

\begin{document}

%%% TITLE
\begin{minipage}[c]{0.5\textwidth}
  \begin{flushleft}\begin{spacing}{1}
    \Huge Ian Dexter D. Marquez

    \Large Linux system administrator
  \end{spacing}\end{flushleft}
\end{minipage}
\begin{minipage}[c]{0.5\textwidth}
  \begin{flushright}\begin{spacing}{1}\small
    +63 917 887 6260 

    \href{mailto:im@iandexter.net}{im@iandexter.net} 

    \href{http://www.iandexter.net}{http://www.iandexter.net} 
  \end{spacing}\end{flushright}
\end{minipage}
\vspace{1em}
\hrulefill
%%% end - TITLE

%%% SUMMARY
With more than 10 years extensive experience in administering Linux and Unix platforms. Develops and maintains various automation tools for infrastructure management and application deployment. Collaborates with global software development and operations teams in application delivery and support. Certified in Red Hat Enterprise Linux, Microsoft Windows, and ITIL Foundation in IT Service Management.
%%% end - SUMMARY

\vspace{1em}

%%% OVERVIEW
\begin{minipage}[t]{0.32\linewidth}
  \begin{section}*{Qualifications}
    \vspace{0.25em}
    \textbf{Bachelor of Science in Computing major in Computer Science}, \textsc{2011}

    Technological University of the Philippines, Manila
    
    \vspace{1em}

    \textbf{Red Hat Certified Engineer}, \textsc{October 2008}

    \textbf{ITIL v3 Foundation in IT Service Management}, \textsc{July 2008}

    \textbf{Red Hat Certified Technician}, \textsc{November 2006}

    \textbf{Microsoft Certified Professional}, \textsc{May 2006}

    \textbf{Trend Micro Certified Security Expert}, \textsc{June 2006}

    \vspace{2em}
    \small
    \begin{subsection}*{Interests}
      \begin{inparadesc}
        \item DevOps,
        \item PaaS, IaaS,
        \item OpenShift, OpenStack, Google App Engine
      \end{inparadesc}
    \end{subsection}
    \vspace{1em}
    \begin{subsection}*{Skill areas}
      \begin{compactitem}
        \item Unix/Linux systems administration (10 years / advanced) 
        \item Infrastructure automation (2 years / proficient) 
        \item Application deployment and support (6 years / advanced) 
        \item Network design, development and administration (5 years / proficient) 
        \item Web design, development and management (3 years / proficient) 
        \item Identity management (2 years / proficient) 
      \end{compactitem}
    \end{subsection}
    \vspace{1em}
    \begin{subsection}*{Languages}
      \begin{compactitem}
        \item Bash/Shell (10 years / advanced) 
        \item Perl (5 years / proficient) 
        \item PHP (3 years / proficient) 
        \item Python (2 years / proficient) 
        \item SQL (2 years / proficient) 
        \item Javascript (3 years / proficient) 
      \end{compactitem}
    \end{subsection}
    \vspace{1em}
    \begin{subsection}*{Platforms}
      \begin{compactitem}
        \item Red Hat Enterprise Linux, CentOS, Fedora (10 years / advanced) 
        \item SuSE Linux Enterprise Server (3 years / advanced) 
        \item Ubuntu, Debian (2 years / proficient) 
        \item Sun Solaris, IBM AIX (1 year / basic) 
        \item Microsoft Windows 2003 Server (1 year / basic) 
      \end{compactitem}
    \end{subsection}
  \end{section}
\end{minipage}
%%% end- OVERVIEW
\hfill
%%% EXPERIENCE
\begin{minipage}[t]{0.65\linewidth}
  \begin{section}*{Experience}
    \begin{subsection}*{Linux system administrator \hfill\textsc{April 2010 to present}}
      \href{http://www.adb.org/}{\textit{Asian Development Bank}} (under RCG IT, Inc.)

      Develops and maintains automation tools for managing the Unix infrastructure for the Bank's mission-critical applications like ERP, data warehousing, financials, and business intelligence. The Asian Development Bank is a multilateral financial institution that provides technical and financial assistance to developing countries.
    \end{subsection}
    \vspace{1em}
  
    \begin{subsection}*{Senior software engineer \hfill\textsc{May 2007 to April 2010}}
      \href{http://www.accenture.com}{\textit{Accenture, Inc.}} 

      Supported the global identity management and web infrastructure of a leading financial services firm. Accenture is an industry leader in global management consulting, technology services, and outsourcing.
    \end{subsection}
    \vspace{1em}
  
    \begin{subsection}*{Unix system administrator \hfill\textsc{January 2007 to April 2007}}
      \href{http://www.adb.org}{\textit{Asian Development Bank}} (under Cytronics, Inc.)

      Operated and maintained Unix servers in the Bank's data center. The Asian Development Bank is a multilateral financial institution that provides technical and financial assistance to developing countries.
    \end{subsection}
    \vspace{1em}
  
    \begin{subsection}*{Systems engineer \hfill\textsc{April 2006 to January 2007}}
      \href{http://www.trendmicro.com}{\textit{Trend Micro, Inc.}} 

      Provided intermediate-level technical and engineering support for Trend Micro enterprise products as part of its EMEA Technical Support Center infrastructure. Trend Micro is a global leader in anti-virus and security solutions, providing comprehensive enterprise-level services across multiple platforms and customer segments.
    \end{subsection}
    \vspace{1em}
  
    \begin{subsection}*{Linux and web administrator \hfill\textsc{July 2004 to March 2006}}
      \href{http://www.philrice.gov.ph}{\textit{Philippine Rice Research Institute}} 

      Managed Linux servers of the corporate intranet in the local and wide area networks, and numerous web-based services for the Institute. Developed a portal for the Open Academy for Philippine Agriculture, the \href{http://www.openacademy.ph}{Pinoy Farmers' Internet}. The Philippine Rice Research Institute is a premier research and development facility in Asia that pioneers and promotes rice science and technology.
    \end{subsection}
    \vspace{1em}
  
    \begin{subsection}*{Network administrator \hfill\textsc{April 2002 to July 2004}}
      \href{http://www.nia.gov.ph}{\textit{National Irrigation Administration}} 

      Designed and implemented the NIA Central Office local area network, and implemented services such as Active Directory, intranet, web, e-mail, and instant messaging. The National Irrigation Administration is the lead Philippine agency that provides irrigation services, development, operation and maintenance.
    \end{subsection}
    \vspace{1em}
  
    \begin{subsection}*{Community relations / information assistant \hfill\textsc{February 1999 to March 2002}}
      \textit{National Irrigation Administration}

      Implemented the public information infrastructure for the Casecnan Multipurpose Irrigation and Power Project (CMIPP), and frequently assigned to IT-related activities. The CMIPP is NIA's flagship project that provide irrigation services to northern Central Luzon and supplementary power to the Luzon grid of the National Power Corporation.
    \end{subsection}
    \vspace{1em}
  \end{section}
\end{minipage}
%%% end - EXPERIENCE

\end{document}