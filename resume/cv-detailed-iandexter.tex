\documentclass[10pt, a4paper, final]{article}

%%% PACKAGES and FORMATTING
\usepackage{marginnote}
\usepackage{geometry}
\geometry{a4paper, margin=2.5cm, vcentering=true, marginparwidth=5cm, marginparsep=1cm}

\usepackage[english]{babel}
\usepackage[UKenglish]{isodate}
\cleanlookdateon

\usepackage[hidelinks, final]{hyperref}
\hypersetup{
  pdfauthor = {Ian Dexter D. Marquez},
  pdftitle = {Curriculum vitae - Ian Dexter D. Marquez},
  pdfsubject = {Generated using Python and LaTeX},
  pdfkeywords = {Amazon Web Services, Ansible, Apache, Application deployment and support, Azure, Bash, Bright Cluster Manager, CentOS, Chef, Cisco IOS, Courier, Debian, DevOps, Docker, Dokuwiki, Dovecot, Drupal, Ethereal/Wireshark, F5 BIG-IP ZebOS, Fedora, Flask, Git, Google Cloud Platform, IBM AIX, IaaS, Identity management, Infrastructure automation, JavaScript, Kubernetes, Laravel, MediaWiki, Memcache, Microsoft Windows 2003 Server, MySQL, Nagios, Netcool, Network design, development and administration, Nginx, Node.js, OpenShift, OpenStack, Oracle, PHP, PaaS, Perl, Postfix, PostgreSQL, Procmail, Python, Red Hat Enterprise Linux, Redis, SQL, Sendmail, Slurm, Sniffer, Squid, SuSE Linux Enterprise Server, Subversion, Sun Directory Server, Sun Identity Manager, Sun Solaris, Sybase, TWiki, Tomcat, Trend Micro InterScan Web and Messaging Security Suite for Windows and Unix, Ubuntu, Unix/Linux systems administration, VMware, VirtualBox, Web design, development and management, WordPress, tcpdump, curriculum vitae, CV, LaTeX, Jinja2}
}

\usepackage[osf]{libertine}
\renewcommand*\familydefault{\sfdefault}
\usepackage[T1]{fontenc}

\usepackage{titlesec}
\titleformat{\section}{\sc\large\center}{}{}{}
\titleformat{\subsection}{\bf\raggedright}{}{}{}
\titlespacing{\section}{0pt}{1em}{1.5em}
\titlespacing{\subsection}{0pt}{0.5em}{0em}

\usepackage{paralist}
\usepackage{setspace}

\usepackage[usenames,dvipsnames]{xcolor}
\definecolor{dark-gray}{gray}{0.30}

\usepackage{fancyhdr}
\usepackage{lastpage}
\fancyhf{}
\renewcommand{\headrulewidth}{0pt}
\renewcommand{\footrulewidth}{0.1pt}
\renewcommand{\footrule}{\hbox to\headwidth{%
  \color{gray}\leaders\hrule height \footrulewidth\hfill}}
\fancyfoot[L]{\small \textcolor{gray}{CV - \textsc{Ian Dexter D. Marquez} (Last update:  \textsc{\today})}}
\fancyfoot[R]{\small \textcolor{gray}{Page \thepage\ of \pageref{LastPage}}}
\pagestyle{fancy}

\setlength{\parskip}{0cm}
\setlength{\parindent}{0cm}
%%% end - PACKAGES and FORMATTING

\begin{document}

%%% TITLE
\begin{center}
  \Huge Ian Dexter D. Marquez\\
  \Large Linux systems administrator
\end{center}
%%% end - TITLE

%%% ADDRESS
\begin{center}\begin{spacing}{1}\small
  +31 6 1197 1691~\textbullet~\href{mailto:im@iandexter.net}{im@iandexter.net}~\textbullet~\href{https://www.iandexter.net}{https://www.iandexter.net}\\
  Faas Wilkestraat 109K, 1095MD Amsterdam, North Holland, The Netherlands\\
  
\end{spacing}\end{center}
\vspace{1.5em}
%%% end - ADDRESS

%%% SUMMARY
\normalsize With more than 10 years extensive experience in administering Linux and Unix platforms. Develops and maintains various automation tools for infrastructure management and application deployment. Collaborates with global software development and operations teams in application delivery and support. Certified in Red Hat Enterprise Linux, Microsoft Windows, and ITIL Foundation in IT Service Management. \small \textcolor{dark-gray}{Last update: \textsc{\today}} \normalsize
%%% end - SUMMARY

\vspace{1em}

%%% OVERVIEW
\begin{section}*{Overview}
  \begin{subsection}*{Skill areas}
    \begin{compactitem}
      \item Unix/Linux systems administration (17 years / advanced) 
      \item Infrastructure automation (9 years / proficient) 
      \item Application deployment and support (12 years / advanced) 
      \item Network design, development and administration (5 years / proficient) 
      \item Web design, development and management (3 years / proficient) 
      \item Identity management (2 years / proficient) 
    \end{compactitem}
  \end{subsection}
\end{section}
%%% end - OVERVIEW

\vspace{1em}

%%% EXPERIENCE
\begin{section}*{Experience}
  \begin{subsection}*{Support and deployment engineer \hfill\textsc{February 2019 to present}}
    \href{https://www.brightcomputing.com/}{\textit{Bright Computing}} \hfill Amsterdam, The Netherlands
    \vspace{1em}

    Supports customers in managing their high-performance clusters (HPC) using Bright Cluster Manager (BCM). Bright Computing develops high-quality software for deploying and managing high-performance clusters, Kubernetes clusters, and OpenStack private clouds in on-premise data centers and public cloud platforms.
    \vspace{1em}
    \begin{compactitem}
      \item Provides intermediate-level support for managing HPCs, workload schedulers, Kubernetes clusters, and OpenStack private clouds.
      \item Deploys high-density HPCs for global customers.
      \item Coordinates closely with software development teams in identifying and resolving issues identified through customer reports or testing.
      \item Works on the latest hardware like GPUs and high-speed interconnects, and software stacks such as distributed filesystems and hypervisors, to leverage the functionalities of the HPC management software.
      \item Evaluates new technologies in the HPC space for integration with BCM.
      
    \end{compactitem}
  \end{subsection}
  \vspace{2.5em}

  \begin{subsection}*{Senior DevOps engineer \hfill\textsc{December 2015 to December 2018}}
    \href{http://www.compareglobalgroup.com}{\textit{CompareGlobalGroup}} \hfill Hong Kong
    \vspace{1em}

    Heads the distributed team that handles the maintenance and operation of the organization's cloud infrastructure. CompareGlobalGroup is the most comprehensive financial comparison platform in the world, with presence in Asia, Europe and Latin America.
    \vspace{1em}
    \begin{compactitem}
      \item Leads a team of system administrators and engineers who manage the infrastructure spread out across multiple availability zones and regions.
      \item As part of the technology management team, assists the CTO in setting the technology direction for the organisation.
      \item Established a service desk for incident management, providing a single point of contact for customers, and greatly improved the response time to production issues.
      \item Led the project to develop and maintain a business monitoring platform, which improved the visibility of application uptime.
      \item Develops and maintains tools for infrastructure management and application deployment.
      \item Provides technical guidance in infrastructure management for other ventures under \href{http://www.novafounders.com/}{Nova Founders Capital}.
      
    \end{compactitem}
  \end{subsection}
  \vspace{2.5em}

  \begin{subsection}*{DevOps engineer (part-time) \hfill\textsc{November 2014 to November 2015}}
    \href{http://www.compareglobalgroup.com}{\textit{CompareGlobalGroup}} \hfill Makati City
    \vspace{1em}

    Part of the global team that created and upgraded the organization's cloud infrastructure. CompareGlobalGroup is the most comprehensive financial comparison platform in the world, with presence in Asia, Europe and Latin America.
    \vspace{1em}
    \begin{compactitem}
      \item Managed the infrastructure spread out across multiple availability zones and regions.
      \item Developed and maintained tools for infrastructure management and application deployment.
      
    \end{compactitem}
  \end{subsection}
  \vspace{2.5em}

  \begin{subsection}*{Linux system administrator \hfill\textsc{April 2010 to December 2015}}
    \href{http://www.adb.org}{\textit{Asian Development Bank}} (under RCG IT, Inc.) \hfill Pasig City
    \vspace{1em}

    Developed and maintained automation tools to manage the Unix infrastructure for the Bank's mission-critical applications like ERP, data warehousing, financials, and business intelligence. The Asian Development Bank is a multilateral financial institution that provides technical and financial assistance to developing countries.
    \vspace{1em}
    \begin{compactitem}
      \item Provided platform-level support to the Bank's development and infrastructure operations teams.
      \item Coordinated with various software development teams on continuous integration, rapid deployment, performance monitoring and tuning, and application support.
      \item Developed and maintained tools for application deployment, monitoring, and performance tuning.
      \item Documented testing and deployment scenarios for the Unix infrastructure requirements of the Bank.
      \item Supported Unix servers that run Apache, Tomcat and WebSphere web servers, Oracle and MySQL database servers, Oracle ERP applications, and IBM business intelligence software.
      \item Maintained an automation script that improves the deployment of Oracle applications. The deployment script, used within the Oracle web UI, was not actively maintained, and had numerous feature requests and issues submitted by the development teams. Upon assumption of code maintenance, feature requests and bugs were reduced to less than 10\%.
      \item Wrote an orchestration tool that provides self-service capabilities for tasks such as database and application server restarts, log retrieval, initiation of cold backups, among others. Commonly requested tasks from the development and operations previously went through the service request queues before being acted on by other teams. The use of the tool significantly decreased these requests, improving workflows across various teams.
      \item Streamlined the provisioning process for Linux (physical and virtual) servers, which reduced build times from one day to less than an hour.
      \item Automated the security baseline configuration for Linux servers as part of compliance requirements at the Bank. Quarterly security audits tended to be time-consuming exercises for administrators. The automation process helped decrease the number of audit findings and interventions to more than 60\%.
      \item Created and deployed a management tool for setting scheduled jobs from a centralized repository. Scheduled jobs were previously stored and managed locally. With the sync tool, jobs were centrally managed from a repository, and pulled by the corresponding servers, thus improving the scheduling workflow, and minimizing errors. This project was accepted in the \href{https://lopsa.org/mentor}{League of Professional System Administrators (LOPSA) Mentorship Program}.
      \item Re-factored a service that routes SMS acknowledgements for Nagios alerts. The previous service was not capable of accepting alert acknowledgements through SMS. The new service resulted in shorter response times from different operations teams when acting on service alerts.
      \item Wrote Nagios plugins and scripts for a monitoring dashboard. The dashboard, used by analysts for a critical business application, provides a single interface for tracking message workflows across various platforms.
      \item Wrote and maintained various scripts in Perl, Python and Bash for automating system administration, performance monitoring, and tuning tasks.
      \item As subject-matter expert, provided quality control and source code review for automation tools and scripts used in the Unix infrastructure.
      
    \end{compactitem}
  \end{subsection}
  \vspace{2.5em}

  \begin{subsection}*{Senior software engineer \hfill\textsc{May 2007 to April 2010}}
    \href{http://www.accenture.com}{\textit{Accenture, Inc.}} \hfill Pasig City
    \vspace{1em}

    Supported the global directory services, identity management, and web infrastructure of a leading financial services firm. Accenture is an industry leader in global management consulting, technology services, and outsourcing.
    \vspace{1em}
    \begin{compactitem}
      \item Maintained the firm's directory services and identity management infrastructure using third-party open source and internally developed tools.
      \item Supported the firm's global web and enterprise products infrastructure.
      \item Worked closely with global software development teams in the areas of application deployment, monitoring and support.
      \item Resolved infrastructure issues reported through automated alerts, and from the global help desk.
      \item Subject-matter expert and training resource for shell scripting and Linux administration.
      \item Created a web application that interfaced with a service fulfilment facility. The application significantly streamlined the creation and closure of service requests for common tasks.
      \item Wrote and maintained various scripts to automate administrative tasks, including a tool for group management notification, a helper script that gathers LDAP information from various sources, and checkout scripts that provide inputs for the global monitoring framework.
      
    \end{compactitem}
  \end{subsection}
  \vspace{2.5em}

  \begin{subsection}*{Unix system administrator \hfill\textsc{January 2007 to April 2007}}
    \href{http://www.adb.org}{\textit{Asian Development Bank}} (under Cytronics, Inc.) \hfill Pasig City
    \vspace{1em}

    Operated and maintained Unix servers in the Bank's data centre. The Asian Development Bank is a multilateral financial institution that provides technical and financial assistance to developing countries.
    \vspace{1em}
    \begin{compactitem}
      \item Ensured 99.999\% uptime of production Unix (Solaris, AIX, and SLES) servers in the Bank's data centre.
      \item Projects included the deployment of Subversion repositories for more than 20 software development projects; server consolidation; staging and installation of centralized log servers for Unix and Windows platforms; and testing and deployment of server backup strategies (on SAN) for about 100 Unix servers.
      
    \end{compactitem}
  \end{subsection}
  \vspace{2.5em}

  \begin{subsection}*{Systems engineer \hfill\textsc{April 2006 to January 2007}}
    \href{http://www.trendmicro.com}{\textit{Trend Micro, Inc.}} \hfill Quezon City
    \vspace{1em}

    Provided intermediate-level technical and engineering support for Trend Micro enterprise products as part of its EMEA Technical Support Centre infrastructure. Trend Micro is a global leader in anti-virus and security solutions, providing comprehensive enterprise-level services across multiple platforms and customer segments.
    \vspace{1em}
    \begin{compactitem}
      \item Handled troubleshooting and technical issue resolution for internet gateway (web and messaging) security products on Windows, Solaris and Linux platforms, using intensive testing procedures and set-up of lab environments for simulations and software patch testing.
      \item Initiated customer recovery procedures through teleconferences and web-based live meetings with senior product specialists and developers.
      \item Performed beta testing and technical documentation of upcoming products such as network appliances and gateway security software.
      \item Acted as a subject-matter resource on Linux and systems administration, and provided technical coaching for new engineers.
      \item Contributed technical solutions articles to Trend Micro's knowledge base.
      
    \end{compactitem}
  \end{subsection}
  \vspace{2.5em}

  \begin{subsection}*{Linux and web administrator \hfill\textsc{July 2004 to March 2006}}
    \href{http://www.philrice.gov.ph}{\textit{Philippine Rice Research Institute}} \hfill Science City of Mu{\~n}oz
    \vspace{1em}

    Managed Linux servers of the corporate intranet in the local and wide area networks, and numerous web-based services for the Institute. Developed a portal for the Open Academy for Philippine Agriculture, the \href{http://www.openacademy.ph}{Pinoy Farmers' Internet}. The Philippine Rice Research Institute is a premier research and development facility in Asia that pioneers and promotes rice science and technology.
    \vspace{1em}
    \begin{compactitem}
      \item Managed Linux systems (on Red Hat, Fedora and CentOS) with services such as DNS, IMAP, LDAP, proxy, database, and HTTP.
      \item Developed and maintained the web portal for the Pinoy Farmers' Internet.
      \item Successfully deployed a 30-node Linux Terminal Server Project (LTSP) site running Ubuntu Linux as the IT training lab for the PhilRice Biotechnology Intellectual Property Rights Center.
      \item Deployed and maintained the Farmers' Contact Center web application in collaboration with the DOST Advanced Science and Technology Institute (ASTI). With the app, farmers and agricultural extension workers can send queries, and receive tips through SMS.
      \item As a science research specialist, performed research on the evaluation and usability of learning management and content delivery systems for agriculture extension workers; and the evaluation and installation of a content management system for the corporate intranet and the web portal. These studies were presented and gained recognition in various scientific research fora.
      \item Other projects: configuration, deployment and maintenance of redundant database servers with high-availability and fail-over; high-availability and load-balancing clusters for WAN routers and local web proxy caches; improved network security through reformulation and implementation of network policies; and migration from proprietary software to open source.
      
    \end{compactitem}
  \end{subsection}
  \vspace{2.5em}

  \begin{subsection}*{Network administrator \hfill\textsc{April 2002 to July 2004}}
    \href{http://www.nia.gov.ph}{\textit{National Irrigation Administration}} \hfill Quezon City
    \vspace{1em}

    Designed and implemented the NIA Central Office local area network, and implemented services such as Active Directory, intranet, web, e-mail, and instant messaging. The National Irrigation Administration is the lead Philippine agency that provides irrigation services, development, operation and maintenance.
    \vspace{1em}
    \begin{compactitem}
      \item Initiated the partnership between the NIA and PREGINET of DOST ASTI for the Agency's inclusion to the nationwide broadband research and education network.
      \item Designed, prepared bid documents and specifications, and managed the structured cabling project of the NIA CO LAN.
      \item Developed the roll-out and implementation strategy for the NIA CO LAN.
      \item Developed and maintained the company intranet using Apache, PHP and MySQL, with a content management system running on Drupal.
      
    \end{compactitem}
  \end{subsection}
  \vspace{2.5em}

  \begin{subsection}*{Community relations / information assistant \hfill\textsc{February 1999 to March 2002}}
    \textit{National Irrigation Administration} \hfill Science City of Mu{\~n}oz
    \vspace{1em}

    Implemented the public information infrastructure for the Casecnan Multipurpose Irrigation and Power Project (CMIPP), and frequently assigned to IT-related activities. The CMIPP is NIA's flagship project that provide irrigation services to northern Central Luzon and supplementary power to the Luzon grid of the National Power Corporation.
    \vspace{1em}
    \begin{compactitem}
      \item Designed and developed information materials such as stand-alone kiosks, multimedia presentations, magazines, newsletters, brochures, flyers, and radio plugs.
      \item Involved in the Project's IT initiatives like software and hardware acquisitions, technical evaluation in IT procurement activities, and maintenance and troubleshooting of PC hardware.
      \item Spearheaded the creation of a Public Information Desk for the Project.
      
    \end{compactitem}
  \end{subsection}
  \vspace{2.5em}
\end{section}
%%% end - EXPERIENCE

%%% OTHERS
\begin{section}*{Other information}
  \begin{inparadesc}
    \item \textbf{Interests}:
    \item DevOps,
    \item Amazon Web Services, OpenShift, OpenStack, Google Cloud Platform,
    \item PaaS, IaaS
  \end{inparadesc}
  \vspace{1em}
  \begin{subsection}*{Qualifications}
    \begin{compactitem}
      \vspace{0.25em}
      \item \textbf{Bachelor of Science in Computing major in Computer Science}, \textsc{2011}\\
      Technological University of the Philippines, Manila
      \vspace{0.25em}
      
      \item \textbf{Red Hat Certified Engineer}, \textsc{October 2008 (not current)}
      \item \textbf{ITIL v3 Foundation in IT Service Management}, \textsc{July 2008}
      \item \textbf{Red Hat Certified Technician}, \textsc{November 2006}
      \item \textbf{Microsoft Certified Professional}, \textsc{May 2006}
      \item \textbf{Trend Micro Certified Security Expert}, \textsc{June 2006}
    \end{compactitem}
  \end{subsection}
  \vspace{1em}
  \begin{subsection}*{Languages}
    \begin{compactitem}
      \item Bash (17 years / advanced) 
      \item Perl (9 years / proficient) 
      \item PHP (3 years / proficient) 
      \item Python (7 years / proficient) 
      \item SQL (2 years / proficient) 
      \item JavaScript (7 years / proficient) 
    \end{compactitem}
  \end{subsection}
  \vspace{1em}
  \begin{subsection}*{Platforms}
    \begin{compactitem}
      \item Red Hat Enterprise Linux, CentOS, Fedora (17 years / advanced) 
      \item SuSE Linux Enterprise Server (3 years / advanced) 
      \item Ubuntu, Debian (7 years / proficient) 
      \item Sun Solaris, IBM AIX (1 year / basic) 
      \item Microsoft Windows 2003 Server (1 year / basic) 
    \end{compactitem}
  \end{subsection}
  \vspace{1em}
  \begin{subsection}*{Tools}
    \begin{compactitem}
      \item Network and systems administration: Ansible, Chef, Nagios, Netcool, Apache, Tomcat, Squid, Nginx, Postfix, Sendmail, Courier, Dovecot, Procmail, MySQL, PostgreSQL, Sybase, Oracle, Redis, Memcache, Trend Micro InterScan Web and Messaging Security Suite for Windows and Unix, Ethereal/Wireshark, Sniffer, tcpdump, Cisco IOS, F5 BIG-IP ZebOS, VMware, VirtualBox, Docker, Kubernetes, Subversion, Git, Amazon Web Services, OpenShift, OpenStack, Google Cloud Platform, Azure
      \item Identity management: Sun Directory Server, Sun Identity Manager
      \item Web and content development: Drupal, WordPress, TWiki, MediaWiki, Dokuwiki, Laravel, Node.js, Flask
      \item High-performance computing: Bright Cluster Manager, Slurm
    \end{compactitem}
  \end{subsection}
\end{section}
%%% end - OTHERS

\vspace{2em}

%%% REFERENCES
\begin{section}*{References}
  \begin{subsection}*{Rhodora Ricafrente Borlas}
    \begin{compactitem}
      \item[] \textit{Unix Implementation Engineer}
      \item[] Bank of America Merrill Lynch, Singapore
      \item[] +65 9002 0397, \href{mailto:rhodora.borlas@baml.com}{rhodora.borlas@baml.com}
    \end{compactitem}
  \end{subsection}

  \begin{subsection}*{Josette Salac}
    \begin{compactitem}
      \item[] \textit{Senior Merchandise Manager}
      \item[] Robinsons Retail Holdings, Inc., Quezon City
      \item[] +63 998 846 6207, \href{mailto:josette.salac@daiso.com.ph}{josette.salac@daiso.com.ph}
    \end{compactitem}
  \end{subsection}

  \begin{subsection}*{Wilmeir Arevalo}
    \begin{compactitem}
      \item[] \textit{Senior Portal Architect}
      \item[] Teradyne Philippines Ltd., Lapu-lapu City
      \item[] +63 932 429 7965, \href{mailto:wilmeir.arevalo@teradyne.com}{wilmeir.arevalo@teradyne.com}
    \end{compactitem}
  \end{subsection}

  \begin{subsection}*{John Paul Villaran}
    \begin{compactitem}
      \item[] \textit{Senior Developer}
      \item[] KMC Solutions, Taguig City
      \item[] +63 998 558 9361, \href{mailto:jp.villaran@gmail.com}{jp.villaran@gmail.com}
    \end{compactitem}
  \end{subsection}
\end{section}
\vspace{1em}
\end{document}
%%% end - REFERENCES